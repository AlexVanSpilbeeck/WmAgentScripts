\documentclass[journal, transmag]{IEEEtran}
%\bibliographystyle{iopart-num}
\bibliographystyle{amsalpha}
\usepackage{graphicx}
\usepackage{url}
\usepackage{cite}

\begin{document}

\title{``The workflows: Simulation and Reprocessing" for the CMS Computing Operations in the LHC run1}

\author{\IEEEauthorblockN{Edgar Fajardo\IEEEauthorrefmark{1},
Jenniffer Adelman-McCarthy\IEEEauthorrefmark{2},
Alan Malta\IEEEauthorrefmark{3},
Diego Ballesteros\IEEEauthorrefmark{4},Xavier Janssen \IEEEauthorrefmark{5}
}
\IEEEauthorblockA{\IEEEauthorrefmark{1}Universidad de los Andes, Colombia}
\IEEEauthorblockA{\IEEEauthorrefmark{2}Fermi National Accelerator Laboratory, USA}
\IEEEauthorblockA{\IEEEauthorrefmark{3}UERJ, Brazil}
\IEEEauthorblockA{\IEEEauthorrefmark{4}Fermi National Accelerator Laboratory, USA}
\IEEEauthorblockA{\IEEEauthorrefmark{5}Universiteit Antwerpen, Belgium}% <-this % stops an unwanted space
\thanks{Reviewer?}}

\IEEEtitleabstractindextext{%
\begin{abstract}
The main objective of the CMS central computing operations is to run and monitor the simulation and data reconstruction workflows maximizing the throughput while making an efficient use of all the computing resources. Running the central workflows presented itself as a changing and challenging task as the conditions, tools  and manpower varied during the first run of data taking.

In this report we present the way in which running this workflows evolved with the needs of the CMS. This will include how we evolved from the first framework tool ProdAgent to the actual WMAgent, the change in the grid infrastructure submission, from GLite to GlideIn and finally the organization of the procedures around the Workflow Team. We will end up with the biggest achievements the team made during the first run of data taking our actual setup and the future challenges that we will meet in the second run as well as the expected solutions.
\end{abstract}

% Note that keywords are not normally used for peerreview papers.
\begin{IEEEkeywords}
Grid, Distributed Computing, MonteCarlo Simulation
\end{IEEEkeywords}}

\maketitle

\section{Introduction}
From 2009 to 2013, the Compact Muon Soleniod (CMS) one of the four detectors at the LHC collected data from proton-proton collisions. One of the main tasks of the CMS computing is to process this data as quickly and reliable as possible, as well as producing the Monte Carlo (MC) simulations needed for the end-user analysis. The task was designed to be performed using the resources described in the MONARC project \cite{Aderholz2000}. Those resources were organized and distributed in a hierachical way around the globe \cite{Bonacorsi2011}.

At the top of hierarchy lies the Tier-0, a unique facility located at CERN. According to the CMS Computing model it is responsible for storing a cold archival copy of the RAW files that are out of the CMS Online Data Acquisition and trigger system. Additionally it is responsible for the repacking and prompy reconstruction of data, classificatom pf data into Primary Datasets (PDs) acoording to their physics conent and disribution of these PDs amond sites at the next Tier level (Tier-1).


\section{Different kind of workflows}

\subsection{Data}
\subsection{MonteCarlo}

\section{How we ran prior to 2011}
\subsection{ProdAgent Vs WMAgent}
\subsection{Reprocessing and Production}

\section{How we ran with WMAgent (after 2011)}
\subsection{WMAgent /ReqMgr/Workqueue (Diego/Edgar/Alan) General comment on how it works}
\subsection{PREP/McM/ReqManager Interaction (Vincenzo?)}
\subsection{Organization of the workflow team and operations around it }

\section{monitoring for T1 and T2 sites}
\section{Achievements}
\subsection{Events reconstructed}
\subsection{Usage of the grid}
\section{Conclusions/Outlook}
\section*{References}
\bibliography{iopart-num}

\end{document}


